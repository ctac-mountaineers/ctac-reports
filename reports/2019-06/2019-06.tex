\documentclass[nonacm,acmtog,authordraft]{acmart}
\usepackage{xcolor}
\usepackage{graphicx}
\usepackage{subcaption}
\newcommand{\todo}[1]{\textcolor{red}{\textbf{TODO}: #1}}
\newcommand{\TODO}{\textbf{\textcolor{red}{TODO}}}


% Based off of Google Drive "Draft 1"

% ==[ Header information ]======================================================

\title{CTAC-2019-06: Attaching to a Multi-Pitch Anchor}
\subtitle{Climbing Technical Advisory Committee}

% ==[ Abstract ]================================================================

\begin{abstract}
  Attaching oneself to a multi-pitch anchor when climbing is a crucial safety
  step and one that sees a wide range of applications in the field across the
  Mountaineers community---at both the basic and intermediate levels. There
  appear to be 2 primary approaches to tying into an anchor that are taught and
  recommended within the Mountaineers climbing programs: attachment with a
  clove hitch only and attachment with both a clove hitch and a backup. With
  the latter approach, there are at least 3 backup methods in practice across
  the branches: a second clove hitch, a Personal Anchor System (PAS), and a
  Figure-8 knot\footnote{The Bellingham branch also teaches the use of the
  Double Figure 8 Loop---or Bunny Ears knot---at the intermediate level. The
  evaluation of this method is out of scope for this report.}.
     
  After reviewing these different practices and reviewing industry standards,
  the consensus on the CTAC committee---in alignment with IMG from their
  internal audit of our climbing programs in 2015---is that the use of a single
  clove hitch tie-in is a safe best practice. An optional backup can be used
  when the climber or belayer feel it is necessary based on the conditions.
\end{abstract}

\begin{document}
\maketitle

\section{Introduction}
  A request was made of the CTAC committee to advise on the optimal method for
  tying into a multi-pitch anchor given “safety, efficiency, anchor management,
  [and] popular/common usage.” As a result of this request, a survey was
  conducted across the CTAC members to determine what is in use today. In
  addition, several online and in-person discussions were held to review these
  practices and determine the CTAC-advised recommendation.

  The CTAC committee also relied on an audit---conducted in 2015 by
  International Mountain Guides (IMG)---of the Mountaineers teaching practices
  in use across the different branches. The IMG report provided a best practice
  on the subject of anchor tie-in which is aligned with CTAC conclusions.
  Specifically, the IMG report states~\cite{ref:imga-course-review}:

\begin{quote}
  If used as a tie in at an anchor (for example, going up on a multi-pitch
  climb) a well tied clove hitch is more than sufficient and backing it up with
  a figure eight only adds clutter and potential confusion to the anchor. While
  this requirement may seem conservative visual clutter can be quite dangerous
  and lead to mistakes as well as slow things down. Consider that the Leader
  should only be tied in with rope. Adding personal anchor (unless it is kept
  slack) removes the added benefit of using the dynamic rope in the system. It
  can also limit the length of adjustability in the system, which can
  compromise comfort and the ability to see and communicate well with the
  follower.
\end{quote}

\section{CTAC Recommendation}
A single tie-in using the rope and clove  is sufficiently strong, reliable,
reduces clutter at the anchor, is efficient, and is aligned with recognized
climbing authorities and guiding companies (i.e. AMGA, AAC, International
Mountain Guides, etc.).  The risk of one climber accidentally unclipping
another climber's anchor is mitigated by instilling a strong culture of
cross-checking and by having climbers weight their tethers.

If there is a real or perceived risk of mechanical or human failure which would
endanger the climbers (i.e.: an incorrectly tied clove hitch), a second tie-in
can be used with that option being another tie-in with the rope either using a
figure 8 or a clove. The least preferred option is to tie-in with a PAS.  The
reasoning is that a rope allows for greater flexibility of positioning the
climber and reduces the risk of falling on a static tether.

Some course leaders will insist on a second tie-in always under all
circumstances for all participants in the class.  In this case, CTAC recommends
that the course explain to the students the reasoning for this requirement and,
as importantly, explain that other courses and the climbing community, in
general, commonly use a single tie-in with the rope and the reasons why. This
sets the larger context for the students so when they get out of the class and
climb with others (either with other Mountaineers, Mountaineers courses, or on
private climber) and they encounter the single rope tie-in, they won't be
confused and alarmed in believing that practice is unsafe.

\bibliographystyle{plain}
\bibliography{../ctac.bib}
\end{document}

% ==[ Workspace ]===============================================================
