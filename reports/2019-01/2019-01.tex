\documentclass[nonacm,acmtog]{acmart}
\usepackage{xcolor}
\usepackage{graphicx}
\usepackage{multirow}
\usepackage{subcaption}

\newcommand{\todo}[1]{\textcolor{red}{\textbf{TODO}: #1}}
\newcommand{\TODO}{\textbf{\textcolor{red}{TODO}}}


\title{CTAC-2019-1: Advice on the Use of Personal Tethers}
\subtitle{Climbing Technical Advisory Committee}

\begin{abstract}
   Personal tethers are a valuable addition to the climbing system.  They offer
   a quick and easy way for climbers to attach themselves to equalized anchors
   or pieces of protection.  However, when used inappropriately or when made of
   inappropriate materials, they can expose the climber to unnecessary risk.
   This report looks at different types of personal tethers, evaluates them
   based on their protection of the climber, and provides useful guidelines on
   how to choose and use tethers in your own climbing.  In general, CTAC finds
   that relying on the rope whenever possible is preferred, but when that's not
   an option, using a separate tether that is dynamic and kept under tension is
   ideal.
\end{abstract}

\begin{document}
\maketitle

\section{Introduction}
\label{sec:intro}

   A personal tether is a piece of material used to connect a climber directly
   to an equalized anchor or piece of protection.  Often, this comes in the form
   of a single- or double-length sling girth-hitched to the harness, or as a
   piece of gear specially made for the purpose, such as the Petzl Connect
   Adjust~\cite{gear:connect} or the Metolius PAS~\cite{gear:pas}.  When used
   appropriately, tethers enable quick and efficient transitions between
   different climbing systems (e.g., from climbing to rappels).  However, when
   used incorrectly or when constructed with inappropriate materials, they can
   expose the climber to unnecessary risk.

   Given the wide range of specialty tethers and materials out there, it can
   sometimes be challenging to find the data needed to make decisions about the
   appropriateness of a particular setup.  In this report, we aim to aggregate
   data from a few different sources, showing the relative strength of each
   material in a number of static and dynamic tests, as well as the ability of
   the materials to absorb the impact of a fall of various fall factors (FF).

   Our goal is to collect this information in a single place, review the
   results, and provide a number of actionable pieces of advice for climbers to
   improve their safety when using a tether.  \S\ref{sec:studies} introduces
   the data used to make our recommendations; \S\ref{sec:guidelines} provides
   three general guidelines for tether use based on this data;
   \S\ref{sec:recommendations} looks at specific products commonly used as
   tethers and provides CTAC's feedback on each; and finally,
   \S\ref{sec:conclusions} summarizes our findings.

\section{Summary of Data Used}
\label{sec:studies}

   Our two primary sources for material strength and dynamic testing are
   reports produced by two prominent climbing equipment manufacturers, Black
   Diamond \cite{bd-pull-tests} and DMM \cite{dmm-pull-tests}.  A summary of
   their results is included in Tables~\ref{tab:bd-results} and
   \ref{tab:dmm-results}, respectively.  For more information on the
   experimental setups and how these results were obtained, please refer to the
   original cited reports cited.  In these tables, all forces are reported in
   kilonewtons (kN).

   Looking first at Black Diamond's results in Table~\ref{tab:bd-results},
   columns 1 and 2 describe the type of material and the specific product that
   tested, column 3 shows the peak force during static pull tests before the
   material failed, columns 4 and 5 show the peak forces measured for different
   dynamic drop tests at FF~0.5 and FF~1, respectively, and column 6 includes any
   additional notes reported by Black Diamond for that particular
   configuration.

   For the DMM results, columns 1 and 2 show the material under test and
   columns 3 and 4 show the peak impact force measured with dynamic drop tests
   at FF~1 and FF~2, respectively.

   \begin{table*}
      \centering
      \Small
      \begin{tabular}{|l|l||c|cc||l|}
      \hline
      \multicolumn{2}{|c||}{\bf Material} &
         {\bf Static} &
         {\bf FF~0.5} &
         {\bf FF~1} &
         {\bf Comments} \\
      \hline
      \multirow{3}{*}{Dyneema/Nylon Mix} &
         BD 60 cm Dynex sling &
         26.1 &
         13.8 &
         20.9 &
         Rated to 22 kN \\
         &
         Metolius PAS &
         22.5 &
         10.9 &
         17.6 &
         Rated to 22 kN \\
         &
         BD Dynex Daisy &
         26.0 &
         2.8 &
         4.9 &
         Ripped anywhere from 5 to 8 pockets during the fall \\
      \hline
      \multirow{2}{*}{Nylon} &
         BD 18mm 60 cm sling &
         26.6 &
         5.3 &
         9.0 &
         \\
         &
         Sterling Chain Reactor &
         17.5 &
         7.9 &
         11.0 &
         \\
      \hline
      \multirow{3}{*}{Rope-based systems} &
         Rope (10.2mm) &
         13.5 &
         3.2 &
         4.6 &
         \\
         &
         Petzl Connect Adjust &
         4.4 &
         4.5 &
         4.8 &
         Started to slip at 4.4 kN during static tests, but did not break \\
         &
         Purcell Prusik (7mm) &
         18.6 &
         3.9 &
         5.3 &
         \\
      \hline
      \end{tabular}
      \caption{Summary of Black Diamond's Static and Dynamic Drop Tests.  See
         \cite{bd-pull-tests} for full test results.}
      \label{tab:bd-results}
   \end{table*}

   \begin{table}
      \centering
      \Small
      \begin{tabular}{|l|l||cc|}
      \hline
      \multicolumn{2}{|c||}{\bf Material} &
         {\bf FF~1} &
         {\bf FF~2} \\
      \hline
      \multirow{3}{*}{Dyneema} &
         8mm sling, single-length (60cm) &
         17.8 &
         22.5* \\
         &
         11mm Dyneema sling, single-length (60cm) &
         16.7 &
         25.1* \\
         &
         11mm Dyneema sling, double-length (120cm) &
         22.4* &
         21.2* \\
      \hline
      \multirow{3}{*}{Nylon} &
         16mm sling, single-length (60cm) &
         11.6 &
         15.4 \\
         &
         26mm sling, single-length (60cm) &
         11.8 &
         16.3 \\
         &
         16mm sling, double-length (120cm) &
         12.8 &
         17.6 \\
      \hline
      \end{tabular}
      \caption{Summary of DMM Drop Tests.  See \cite{dmm-pull-tests} for full
         results.  A * indicates the test configuration failed.}
      \label{tab:dmm-results}
   \end{table}

   Aside from the material strength data discussed above, CTAC also considered
   standards from the Occupational Safety and Health
   Administration (OSHA)~\cite{ref:osha} for the maximum acceptable impact force
   workers at height.  Although the OSHA standards are written in the context
   of occupational protection and do not, in general, apply directly to the
   world of recreational climbing, CTAC found it was useful in this particular
   case to use their recommendations regarding peak impact forces on the human
   body, which were chosen to reflect the ability of the human anatomy to
   safely dissipate impact forces.  In particular, OSHA recommends the peak
   impact force experienced by a person using personal protection equipment is
   limited to around 8 kN of force~\cite[\S1910.140(d)(1)(i)]{ref:osha-peak}.

\section{Guidelines for Tether Use}
\label{sec:guidelines}

   Based on the data from \S\ref{sec:studies}, here we provide a few general
   recommendations on the use of personal tethers and establish some guidelines
   to help decide which types of personal tethers to use.  In
   \S\ref{sec:recommendations}, these guidelines will be used to make specific
   recommendations for different types of tethers.

\subsection{Limit Peak Impact Force to 8 kN}

   When choosing a personal tether system, CTAC recommends that the chosen
   system minimizes the peak impact forces experienced by climbers, ideally
   limiting the force to 8 kN or less.  This number comes directly from the
   OSHA regulations cited previously.

   The peak impact force is a function of both the tether material, as well as
   how the tether is used (more details on this below).  Climbers should make
   sure to use their tether in a way that limit forces they are likely to
   encounter, such as by keeping slack out of the system and staying below the
   tether's attachment point to the anchor.

\subsection{Keep the System Dynamic}
\label{sec:forces}

   As evidenced by Tables~\ref{tab:bd-results} and~\ref{tab:dmm-results}, one of
   the simplest ways to limit peak impact forces is to use a dynamic material
   for the tether.  Both the fully dynamic rope-based system and the semi-static
   nylon-based systems do a good job of keeping peak forces around 8 kN during
   dynamic drop tests of up to FF~1, though the nylon-based materials typically
   produced a slightly higher limit when compared to the rope based systems.
   Additionally, the static pull tests show that the materials themselves are
   capable of holding much greater forces, far exceeding the typical forces seen
   in climbing scenarios.

   As a point of contrast, consider the static materials in these tables, such
   as the Dyneema/Spectra.  Even at relatively small fall factors of 0.5, some
   of these materials were producing forces in excess of 10~kN---enough to
   cause injury to the climber---even though the overall system did not fail.

   Based on these results, it's clear that dynamic materials do a much better
   job of dissipating the fall forces and keeping the peak forces experienced by
   the climber below the previously defined threshold.

%\subsection{Use the Rope When Possible}
%
%   Given that dynamic materials should be preferred over static materials for
%   tethers, it should be clear that the rope is a likely candidate for personal
%   tether material.  In fact, since climbers are tied in to the climbing rope
%   for the entire ascent, using the climbing rope is often the fastest and
%   (according to the results in Table~\ref{tab:bd-results}) the strongest way
%   to secure yourself to the anchor.
%
%   As we'll discuss in \S\ref{sec:recommendations}, CTAC recommends using the
%   climbing %rope as the personal tether whenever possible.

\subsection{Keep the System Weighted}

   Another observation we can make from the data in Tables~\ref{tab:bd-results}
   and~\ref{tab:dmm-results} is that impact forces rise fairly quickly as the
   fall factor increases.  As one concrete example, the Metolius PAS registered
   a peak impact force of 10.9~kN with a FF~0.5 drop test, whereas a FF~1 drop
   produced 17.6~kN, well beyond the safe limits for the human body.  As a
   result, we'd like to ensure that whatever system we use, we use it in a way
   that limits our exposure to these higher fall factors.

   One technique to doing this is to ensure we keep our tether weighted whenever
   possible.  If the tether system is kept weighted at all times, then falls
   are prevented entirely.  If it is impossible to keep the system under
   tension, then taking other steps to reduce the risk of FF~1 or higher falls
   is critical.  One way this can be done, for example, by ensuring the master
   point of the anchor is above the climber's waist.

   Since the location of the anchor is not always controllable, and because
   climber's may need to move around once they have been secured to the anchor
   with a tether, having an adjustable tether that can accommodate a variety of
   tether lengths will greatly improve the climber's ability to keep the system
   weighted, even with these variations.  Because of this, CTAC finds that
   adjustable tethers are generally favored over fixed-length tethers.

\section{Specific Tether Recommendations}
\label{sec:recommendations}

   This section uses the guidelines established in \S\ref{sec:guidelines} to
   make recommendations on specific types of personal tethers.  These
   recommendations are summarized in Table~\ref{tab:summary}, and discussed in
   more detail below.
	
	\begin{table}
	\Small
	\centering
	\begin{tabular}{|r||c|}
		\hline
		\multicolumn{1}{|c||}{\bf Type of tether} & {\bf Recommended?} \\
		\hline
		Climbing Rope & Yes (preferred) \\
		Rope-based tethers (e.g., Connect Adjust) & Yes, (preferred)  \\
		\hline
		Sterling Rope Chain Reactor & Yes \\
		Metolius PAS & Yes \\
		Nylon Slings & Yes \\
		\hline
		Daisy Chains & No \\
		Spectra/Dyneema Slings & No \\
		\hline
	\end{tabular}
	\caption{Summary of tether recommendations}
	\label{tab:summary}
\end{table}


\subsection{CTAC Preferred Personal Tethers}
\label{sec:recommended}

   The tethers in described in this section are the preferred systems by CTAC.
   They do the best job of meeting the guidelines set forth in
   \S\ref{sec:guidelines} and should be used whenever possible.

\subsubsection{Climbing Rope}

   The climbing rope is the single strongest piece of climbing gear we take on
   the wall, and is dynamic by design.  When leading routes, climbers
   implicitly trust that climbing rope can repeatedly catch large falls without
   causing undue harm to the climbers or equipment.

   CTAC recommends using the rope as a tether whenever possible.  It's fast,
   strong, dynamic, and adjustable (with a clove hitch).  The data from Black
   Diamond shows that the climbing rope has the lowest measured impact force of
   all options for FF~1 (4.6~kN), well below the OSHA recommended limit of 8~kN.
   And you're already tied in to it on the ascent!

   Unfortunately, the climbing rope isn't always available as a personal tether
   option: one example where this is the case is during rappels where the
   climber is no longer tied in to the end of the rope.  In these cases, one of
   the other tethers described in this section should be used instead.

\subsubsection{Petzl Connect Adjust (or similar)}

   The Petzl Connect Adjust is a manufactured tether made from 9.2mm climbing
   rope.  Since its made from a climbing rope, it has all of the advantages
   described in the previous section of using the rope directly; however since
   it is an independent system from the main climbing rope, it can be used even
   during the descent.  Other rope-based systems produced by different
   manufactures are likely to be similar.

   The primary downside of rope-based tethers are the relative bulk when
   compared to the alternatives described below; keeping a short-section of rope
   attached to your harness during the ascent can be awkward for climbers,
   especially when leading routes with already-crowded gear loops.  However,
   this disadvantage can be minimized by only relying on the rope-based tether
   during the descent, when the majority of climbing protection can be removed
   from the harness.  During the ascent, CTAC recommends using the main climbing
   rope directly.

\subsection{Other Recommended Personal Tethers}

   The tethers described here are also recommended by CTAC, but generally
   require more careful attention on the part of the climber to be used
   correctly.  For example, both options discussed here are constructed from
   semi-static materials, which can result in higher impact forces if used
   incorrectly.

\subsubsection{Sterling Rope Chain Reactor / Metolius PAS}

   The Chain Reactor and PAS are each constructed of a series of interlocking
   loops.  The chain reactor is primarily a Nylon-based tether, whereas the PAS
   is a Dyneema/Nylon mix.

   Since each loops on these tethers is fully-rated, the loops make it
   easy to adjust the length of the tether to match the position of the anchor,
   allowing the climber to ensure the tether remains weighted.  Additionally,
   the loops can also be used as a convenient rappel extension.  These systems
   are very popular in the recreational climbing community, due to their
   simplicity and availability in many popular retail stores.

   Though the Nylon-only Chain Reactor has lower peak impact forces than the
   Dyneema/Nylon PAS, both resulted in fairly high forces during Black
   Diamond's drop tests.  Both solutions exceeded our target of 8 kN peak with
   FF~1 drops.  The PAS exceeded our target even at FF 0.5 drops, though the
   Chain Reactor was just barely under the bar.

   Due to the adjustability of these options, however, when used appropriately
   and kept under tensions at the anchors, these are considered viable tether
   options by CTAC.  However, for beginning climbers, a dynamic rope-based
   system may be preferred as it can offer increased protection against
   accidental falls.

\subsubsection{Nylon Slings}

   Nylon slings offer a simple, semi-static tether.  They are not quite as
   dynamic as climbing-rope based systems, but they are more dynamic then the
   Dyneema/Spectra materials. As such, their peak impact forces tend to be
   between those two extremes as well.

   When used appropriately, where the tether is an appropriate length and
   clipped above the climber's waist, impact forces for falls were well within
   material-strength and human body limits.  Black Diamond's results showed
   forces that were less than 6 kN with FF 0.5 falls.  However, at higher
   drops, impact forces jumped to 11-12 kN at FF 1 and 15-17 kN at FF 2.
   Though this is within the boundaries of material strength (the Nylon slings
   did not fail in either set of tests), the high impact forces would likely be
   quite dangerous to the climber.

   As such, using Nylon slings as personal tethers requires strict adherence to
   the previous suggestions of keeping the system under tension or otherwise
   minimizing the likelihood of falls.

\subsection{Not Recommended Personal Tethers}
\label{sec:notrecommended}

   Based on CTAC's review, the tethers described here should not be used by
   climbers.  They expose the climber to too much unnecessary risk, especially
   given the number of alternatives on the market (see previous sections).

\subsubsection{Daisy Chains}

   Daisy chains, which are slings that have a number of ``pockets'' stitched
   along the length of the sling, were designed for use during aid climbing and
   were never meant to be used as a personal tether.  Though they seem like a
   convenient and adjustable tether, their construction makes it easy for a
   climber to inadvertently clip across a single, non-load-bearing tack, which
   can fail in even short falls.  This will result in catastrophic failure for
   the climber.

   Additionally, these pockets are not load bearing and will fail at relatively
   low forces (Table~\ref{tab:bd-results}), leaving the daisy destroyed as a
   result.  The previous results showed that even at a modest 0.5 FF drop, 5 of
   the 8 pockets were torn.

   Because of ease of mis-rigging, the one-time use of the daisy, and the wide
   availability of other, more versatile options, CTAC does not recommend the
   use of daisy chains in any capacity outside of traditional aid climbing.

\subsubsection{Spectra or Dyneema Slings}

   Climbers often have many single- and double-length Dyneema/Spectra slings on
   their harness.  They are a light-weight piece of material that can be used to
   construct anchors, equalize protection, and construct extendable quick draws.
   However, these materials are not dynamic, with very little to no extension
   when placed under load.

   Based on the forces highlighted previously both Black Diamond's and DMM's
   drop test results, these materials can put an excessively large force on the
   climber, the anchor, and the sling itself.  Even relatively short falls can
   generate enough force for any one of those components to fail
   catastrophically, as evidenced by the failures in DMM's drop tests.

   Because of the high risk of failure, lack of adjustability, and the wide
   availability of better options, CTAC does not recommend the use of these
   materials as a personal tether.

\section{Conclusions}
\label{sec:conclusions}

   This report consolidated results from two different drop tests completed by
   well-known climbing gear manufacturers.  Based on their results, CTAC made a
   number general recommendations on the appropriate use of tethers in the
   climbing system, and specific recommendations of particular personal tether
   material and products.  In general, keeping the personal tether dynamic and
   under tension is the best way to mitigate the risk of high impact falls and
   equipment failure.

\bibliographystyle{plain}
\bibliography{../ctac.bib}
\end{document}
